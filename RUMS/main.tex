% -----------------------
% preamble
% -----------------------
% Don't change preamble code yourself. If you add something(usepackage, newtheorem, newcommand, renewcommand),
% please tell them editor of institutional paper of RUMS.

%documentclass
%------------------------
\documentclass[11pt, a4paper, dvipdfmx]{jsarticle}


%usepackage
%------------------------
\usepackage{amsmath}
\usepackage{amsthm}
\usepackage[psamsfonts]{amssymb}
\usepackage{color}
\usepackage{ascmac}
\usepackage{amsfonts}
\usepackage{mathrsfs}
\usepackage{amssymb}
\usepackage{graphicx}
\usepackage{fancybox}
\usepackage{enumerate}
\usepackage{verbatim}
\usepackage{subfigure}
\usepackage{proof}
\usepackage{listings}
\usepackage{otf}
\usepackage{algorithm}
\usepackage{algorithmic}
\usepackage{tikz}
\usepackage[all]{xy}
\usepackage{amscd}
\usetikzlibrary{cd}

%theoremstyle
%--------------------------
\theoremstyle{definition}


%newtheoem
%--------------------------
%If you want to use theorem environment in Japanece, You can use these code. 
%Attention
%--------------------------
%all theorem enivironment number depend on only section number.
\newtheorem{Axiom}{公理}[section]
\newtheorem{Definition}[Axiom]{定義}
\newtheorem{Theorem}[Axiom]{定理}
\newtheorem{Proposition}[Axiom]{命題}
\newtheorem{Lemma}[Axiom]{補題}
\newtheorem{Corollary}[Axiom]{系}
\newtheorem{Example}[Axiom]{例}
\newtheorem{Claim}[Axiom]{主張}
\newtheorem{Property}[Axiom]{性質}
\newtheorem{Attention}[Axiom]{注意}
\newtheorem{Question}[Axiom]{問}
\newtheorem{Problem}[Axiom]{問題}
\newtheorem{Consideration}[Axiom]{考察}
\newtheorem{Alert}[Axiom]{警告}


%----------------------------
%If you want to use theorem environment with no number in Japanese, You can use these code.
\newtheorem*{Axiom*}{公理}
\newtheorem*{Definition*}{定義}
\newtheorem*{Theorem*}{定理}
\newtheorem*{Proposition*}{命題}
\newtheorem*{Lemma*}{補題}
\newtheorem*{Example*}{例}
\newtheorem*{Corollary*}{系}
\newtheorem*{Claim*}{主張}
\newtheorem*{Property*}{性質}
\newtheorem*{Attention*}{注意}
\newtheorem*{Question*}{問}
\newtheorem*{Problem*}{問題}
\newtheorem*{Consideration*}{考察}
\newtheorem*{Alert*}{警告}

%--------------------------
%If you want to use theorem environment in English, You can use these code.
%--------------------------
%all theorem enivironment number depend on only section number.
\newtheorem{Axiom+}{Axiom}[section]
\newtheorem{Definition+}[Axiom+]{Definition}
\newtheorem{Theorem+}[Axiom+]{Theorem}
\newtheorem{Proposition+}[Axiom+]{Proposition}
\newtheorem{Lemma+}[Axiom+]{Lemma}
\newtheorem{Example+}[Axiom+]{Example}
\newtheorem{Corollary+}[Axiom+]{Corollary}
\newtheorem{Claim+}[Axiom+]{Claim}
\newtheorem{Property+}[Axiom+]{Property}
\newtheorem{Attention+}[Axiom+]{Attention}
\newtheorem{Question+}[Axiom+]{Question}
\newtheorem{Problem+}[Axiom+]{Problem}
\newtheorem{Consideration+}[Axiom+]{Consideration}
\newtheorem{Alert+}{Alert}

%commmand
%----------------------------
\newcommand{\N}{\mathbb{N}}
\newcommand{\Z}{\mathbb{Z}}
\newcommand{\Q}{\mathbb{Q}}
\newcommand{\R}{\mathbb{R}}
\newcommand{\C}{\mathbb{C}}
\newcommand{\F}{\mathcal{F}}
\newcommand{\X}{\mathcal{X}}
\newcommand{\Y}{\mathcal{Y}}
\newcommand{\Hil}{\mathcal{H}}
\newcommand{\RKHS}{\Hil_{k}}
\newcommand{\Loss}{\mathcal{L}_{D}}
\newcommand{\MLsp}{(\X, \Y, D, \Hil, \Loss)}
\newcommand{\p}{\partial}
\newcommand{\h}{\mathscr}
\newcommand{\mcal}{\mathcal}
\newcommand{\lan}{\langle}
\newcommand{\ran}{\rangle}
\newcommand{\pal}{\parallel}
\newcommand{\dip}{\displaystyle}
\newcommand{\e}{\varepsilon}
\newcommand{\dl}{\delta}
\newcommand{\pphi}{\varphi}
\newcommand{\ti}{\tilde}

\newcommand{\argmax}{\mathop{\rm arg~max}\limits}
\newcommand{\argmin}{\mathop{\rm arg~min}\limits}

\renewcommand{\P}{\mathbb{P}}
\newcommand{\Probsp}{(\Omega, \F, \P)}

%new definition macro
%-------------------------
\def\inner<#1>{\langle #1 \rangle}
%----------------------------
%documenet 
%----------------------------
% Your main code must be written between begin document and end document.
\title{Generative Adversarial Networks}
\author{小泉 孝弥}
\date{}
\begin{document}
\maketitle
\section{Intoroduction}
近年, 機械学習
\section{Fundermetal concepts of Machine Learning}
この節では機械学習の数学的な定式化および, 機械学習の具体例を紹介する. $\X$と$\Y$を線型空間とし, $\Probsp$を
完備な確率空間とする.
\begin{Definition+}(仮説空間, 仮説)\\
    $\X$から$\Y$へのなんらかの条件を満たす写像の集まりのことを仮説空間
    といい$\Hil$と表記する. すなわち,
    \begin{align*}
        \Hil = \{f:\X\to\Y~| f\text{ が満たす条件}\}
    \end{align*}
    である. (条件の具体例は後に述べる). 仮設空間$\Hil$の元のことを仮説と呼ぶ.
    また, この時の$\X$を入力空間, $\Y$を出力空間と呼ぶ. 
\end{Definition+}
これ以降, 入力空間を$\R^{d}$出力空間を$\R^{m}$とする.
\begin{Definition+}(データ)\\
    $\X$と$\Y$の直積集合$\X\times\Y$の有限部分集合のことを教師ありデータという.
    また, $\X$の有限部分集合のことを教師なしデータと呼ぶ. 
\end{Definition+}
\begin{Definition+}(予測損失)\\
    $\Hil$を仮説空間とし, $(X, Y)$をデータの分布$\mathbb{P}_{\X\times\Y}$に従う確率変数と
    する. 以下で定義される$\ell:\Hil\to\R$を予測損失と呼ぶ.
    \begin{align*}
        \ell(f) = \mathbb{E}[\|f(X) - Y\|].
    \end{align*}
\end{Definition+}
予測損失が最小となるような$f\in\Hil$を求めることが機械学習の目標である. しかし, 一般にデータの分布は未知であるため
この式を解くことができない. そこで、損失関数というものを定義し、それを最適化することを考える.
\begin{Definition+}(損失関数)\\
    $\Hil$を仮説空間とする. $\Hil$から$\R$への写像$\Loss:\Hil\to\R$を損失関数(Loss function)と呼ぶ.
\end{Definition+}
\begin{Definition+}(機械学習空間)\\
    $D$をデータ, $\Hil$を仮説空間, $\Loss$を損失関数とする. 
    この時5つ組$\MLsp$を機械学習空間(Machine Learning space)という.
\end{Definition+}
\begin{Definition+}(学習, 最適仮説)\\
    $\Hil$を仮設空間, $\Loss:\Hil\to\R$を損失関数とする. 損失関数が最大または, 最小となるような
    $f^*\in\Hil$を求めること\footnote{機械学習においては厳密に損失関数が最大・最小となる関数が学習されるとは限らないが, そのような関数を求めることも学習の定義に含めるものとする. }を学習といい, $f^*\in\Hil$を最適仮説と呼ぶ.
\end{Definition+}
\begin{Definition+}(機械学習)\\
    機械学習空間$\MLsp$上での学習を機械学習という. 特に, $D$が教師ありデータの時教師あり機械学習と呼び, 
    教師なしデータの時, 教師なし機械学習と呼ぶ.
\end{Definition+}

\section{教師あり機械学習の具体例}
前節では, 機械学習の抽象的な枠組みを紹介したが, この説では機械学習空間の具体例を述べる.
$D = \{(x_i, y_i)\}_{i = 1}^{N}\subset\X\times\Y$を教師ありデータとする.

\subsection{単回帰分析・重回帰分析}
最初に機械学習の最も基本的なモデルである単回帰分析を紹介する。
\begin{Example+}(単回帰分析)\\
    機械学習空間$\MLsp$を以下のように定義する.\\
    $\X = \R$, $\Y = \R$, 
    \begin{align*}
        \Hil = \{f:\X\to\Y~|f(x) = wx, w\in\R\},
    \end{align*}
    \begin{align*}
        \Loss(f) = \sum_{i = 1}^{N}|f(x_i) - y_i|^2.
    \end{align*}
    この機械学習空間$\MLsp$上で,
    \begin{align*}
        \argmin_{f\in\Hil}\Loss(f)
    \end{align*}
    を解く問題を単回帰分析という. 単回帰分析の最適仮説$f^{*}\in\Hil$は
    \begin{align*}
        f^{*}(w) = \frac{\sum_{i = 1}^{N}x_{i}^2}{\sum_{i = 1}^{N}x_{i}y_{i}}w
    \end{align*}
    となる. また, $\Hil\simeq\R$である.
\end{Example+}
\begin{Example+}(重回帰分析)
    
\end{Example+}
\section{Deep Learning}
\subsection{Neural Network}
\subsection{The Universal Theorem of Neural Network}
\section{Generative Adversarial Networks}
\subsection{GANの定式化}
$\Probsp$を完備な確率空間とする.  
$\X$, $\Y$を線型空間とする. 
\begin{align*}
    \Hil_1 &= \{G:\mathcal{Z}\to\X~| G\text{はニューラルネット}\},\\
    \Hil_{2} &= \{D:\X\to [0, 1]~| D\text{はニューラルネット}\}
\end{align*}
ここで、$\mathcal{Z}$は潜在空間と呼ばれる$\R^{\dim\X}$の線型部分空間である.
また, 確率変数$X:\Omega\to\mathcal{Z}$に対し, $g(Z)$が従う確率分布を$\mathbb{P}_{}$
\section{Applications of GANs}

\begin{thebibliography}{20}
    \bibitem{GAN} Ian J. Goodfellow, Jean Pouget-Abadie, Mehdi Mirza, Bing Xu, David Warde- Farley, Sherjil Ozair, Aaron Courville, and Yoshua Bengio. 
    Generative adversarial nets. In Advances in Neural Information Processing Systems, 2014.
    \bibitem{SGAN}Takeru Miyato, Toshiki Kataoka, Masanori Koyama and Yuichi Yoshida, Spectral Normalization for Generative Adversarial Networks, 
    International Conference on Learning Representations, 2018.
\end{thebibliography}

\end{document}