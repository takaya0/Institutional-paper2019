\documentclass[11pt, a4paper, dvipdfmx]{jsarticle}
\usepackage{amsmath}
\usepackage{amsthm}
\usepackage[psamsfonts]{amssymb}
\usepackage{color}
\usepackage{ascmac}
\usepackage{amsfonts}
\usepackage{mathrsfs}
\usepackage{amssymb}
\usepackage{graphicx}
\usepackage{fancybox}
\usepackage{enumerate}
\usepackage{verbatim}
\usepackage{subfigure}
\usepackage{proof}
\usepackage{listings}
\usepackage{otf}

\theoremstyle{definition}

%
%%%%%%%%%%%%%%%%%%%%%%
%ここにないパッケージを入れる人は,必ずここに記載すること.
%
%%%%%%%%%%%%%%%%%%%%%%
%ここからはコード表です.
%



%%%%%%%%%%%%%%%%%%%%

\newtheorem*{Axiom*}{公理}
\newtheorem*{Definition*}{定義}
\newtheorem*{Theorem*}{Theorem}
\newtheorem*{Proposition*}{命題}
\newtheorem*{Lemma*}{補題}
\newtheorem*{Example*}{例}
\newtheorem*{Corollary*}{系}
\newtheorem*{Claim*}{主張}
\newtheorem*{Property*}{性質}
\newtheorem*{Attention*}{注意}
\newtheorem*{Question*}{問}
\newtheorem*{Problem*}{問題}
\newtheorem*{Consideration*}{考察}
\newtheorem*{Alert*}{警告}
\renewcommand{\proofname}{\bfseries Proof}

%%%%%%%%%%%%%%%%%%%%%%
%%

\newcommand{\A}{\bf 証明}
\newcommand{\B}{\it Proof}

%英語で定義や定理を書きたい場合こっちのコードを使うこと.

\newtheorem{Axiom+}{Axiom}[section]
\newtheorem{Definition+}[Axiom+]{Definition}
\newtheorem{Theorem+}[Axiom+]{Theorem}
\newtheorem{Proposition+}[Axiom+]{Proposition}
\newtheorem{Lemma+}[Axiom+]{Lemma}
\newtheorem{Example+}[Axiom+]{Example}
\newtheorem{Corollary+}[Axiom+]{Corollary}
\newtheorem{Claim+}[Axiom+]{Claim}
\newtheorem{Property+}[Axiom+]{Property}
\newtheorem{Attention+}[Axiom+]{Attention}
\newtheorem{Question+}[Axiom+]{Question}
\newtheorem{Problem+}[Axiom+]{Problem}
\newtheorem{Consideration+}[Axiom+]{Consideration}
\newtheorem{Alert+}{Alert}

%commmand

\newcommand{\N}{\mathbb{N}}
\newcommand{\Z}{\mathbb{Z}}
\newcommand{\Q}{\mathbb{Q}}
\newcommand{\R}{\mathbb{R}}
\newcommand{\C}{\mathbb{C}}
\newcommand{\W}{{\cal W}}
\newcommand{\cS}{{\cal S}}
\newcommand{\Wpm}{W^{\pm}}
\newcommand{\Wp}{W^{+}}
\newcommand{\Wm}{W^{-}}
\newcommand{\p}{\partial}
\newcommand{\Dx}{D_{x}}
\newcommand{\Dxi}{D_{\xi}}
\newcommand{\lan}{\langle}
\newcommand{\ran}{\rangle}
\newcommand{\pal}{\parallel}
\newcommand{\dip}{\displaystyle}
\newcommand{\e}{\varepsilon}
\newcommand{\dl}{\delta}
\newcommand{\pphi}{\varphi}
\newcommand{\ti}{\tilde}
\newcommand{\X}{\mathcal{X}}
\newcommand{\Y}{\mathcal{Y}}
\newcommand{\K}{\mathbb{K}}

\title{Universal Approximation Theorem of Neural Network}
\author{数理科学科 3回生 小泉 孝弥}
\date{}
\begin{document}
\maketitle
\begin{abstract}
    機械学習という単語を一言で説明するならば, 「関数近似器」である.
    今回のReMakersの機関紙では機械学習および第3次人工知能ブームの火付け役となった
    「深層学習」を数学的に定義し, 深層学習の中心となる「ニューラルネットワーク」が万能関数近似器であることを示す.
    なお, 本機関紙の前提知識は数学科3回生レベルの数学である.
\end{abstract}
\section{人口知能・AIについて}
現在, 人工知能(Artificial Intelligence)の厳密な定義は完成しておらず, 専門家の中でも意見が分かれている。
なので、この機関紙では人工知能の定義について触れることはせず、現在の, 人工知能の中心技術である「機械学習」について
説明する.
\subsection{機械学習・深層学習}
最初に述べたとおり, 機械学習(Machine Learning)とは関数近似器である. 
もう少し詳細に述べれば, 入力空間と呼ばれる集合$\X$から出力空間と呼ばれる集合への良い写像$f:\X\to\Y$
を構成するのが機械学習である. 機械学習は主に, 株価などのあるもの値を予想するモデルである「回帰(Regression)」, 
あるものがどのクラスに属しているかを予測するモデルである「分類(Classify)」の2つに分けられる.「機械学習の中でも特にあとで定義する, ニューラルネットワークと
いうものを用いる学習手法を深層学習(Deep Learning)という.
\subsection{深層学習と万能近似性}
深層学習は以下の定理から万能関数近似器と呼ばれている. 
\begin{Definition*}(万能関数近似定理)\\
    $X$を集合とし$f:X\to\R$を性質の良い関数とする. この時, $f$を
    任意の精度で近似できるようなニューラルネットワークが存在する.
\end{Definition*}
今回の機関紙の目標はこの定理を証明することであるが, 今のままでは曖昧な単語が多すぎるため, 数学の問題として
扱うことができない. したがって, 今からこの定理を数学的に述べなければならない. しかしながら, 流れの都合上
この定理を数学的に述べる前に, この定理の照明に不可欠な「ハーン・バナッハの拡張定理」と「リースの表現定理」
を証明することにする.
\section{関数解析学}
リースの表現定理およびハーンバナッハの拡張定理は共に関数解析学(Functional Analysis)
と呼ばれる分野の定理である. したがって, この説では関数解析学の基礎事項を説明する. なお, 線形空間の
係数体は実数体または複素数体とし, $\K$で表記する.
\end{document}
